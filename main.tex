\documentclass[11pt]{template/cauchy}
\usepackage[czech]{babel}

% IMPORTY
\usepackage{enumerate}
\usepackage{mlmodern}
\usepackage{template/mathsphystools}
\usepackage{amsthm,thmtools,xcolor,amsmath,amssymb, mathtools}
\usepackage[scr]{rsfso}
% \usepackage{thmstyles}
\usepackage{tabularx}
\usepackage{graphicx}
\usepackage{multicol}
\usepackage{appendix}
\usepackage{tabularx}
\usepackage{subcaption}
\usepackage{hyperref}
\usepackage{comment}
\usepackage{lscape}
\usepackage{xcolor}
\usepackage{wrapfig}
\usepackage{centernot}
\usepackage{pgfplots}
\usepackage{caption}
\graphicspath{{images/}}
\pgfplotsset{width=10cm,compat=1.9}

% \usepackage{tikzit}
% \input{default.tikzstyles}
% je potřeba přiložit soubor default.tikzstyles


% STYLY
% definice, příklad, cvičení, věta, lemma, řešení, poznámka, důsledek, axiom, motivace

\declaretheoremstyle[
  headfont=\color{red}\normalfont\bfseries,
  bodyfont=\color{black}\normalfont,
]{def}

\declaretheoremstyle[
  headfont=\color{blue}\normalfont\bfseries,
  bodyfont=\color{black}\normalfont,
]{pr}

\declaretheoremstyle[
  headfont=\color{teal}\normalfont\bfseries,
  bodyfont=\color{black}\normalfont\itshape,
]{cv}

\declaretheoremstyle[
  headfont=\color{green}\normalfont\bfseries,
  bodyfont=\color{black}\normalfont\itshape,
]{veta}

\declaretheoremstyle[
  headfont=\color{lime}\normalfont\bfseries,
  bodyfont=\color{black}\normalfont\itshape,
]{lemma}

\declaretheoremstyle[
  headfont=\color{black}\normalfont\itshape,
  bodyfont=\color{black}\normalfont,
  numbered=no
]{res}

\declaretheoremstyle[
  headfont=\color{brown}\normalfont\bfseries,
  bodyfont=\color{black}\normalfont,
]{pozn}

\declaretheoremstyle[
  headfont=\color{brown}\normalfont\bfseries,
  bodyfont=\color{black}\normalfont,
]{dusl}

\declaretheoremstyle[
  headfont=\color{cyan}\normalfont\bfseries,
  bodyfont=\color{black}\normalfont,
]{axiom}

\declaretheoremstyle[
  headfont=\color{black}\normalfont\bfseries,
  bodyfont=\color{black}\normalfont,
]{motivace}

\declaretheorem[
  style=def,
  name=Definice,
]{definice}

\declaretheorem[
  style=pr,
  name=Příklad,
]{priklad}

\declaretheorem[
  style=cv,
  name=Cvičení,
]{cviceni}

\declaretheorem[
  style=res,
  name=Řešení,
]{reseni}

\declaretheorem[
  style=veta,
  name=Věta,
]{veta}

\declaretheorem[
  style=lemma,
  name=Lemma,
]{lemma}

\declaretheorem[
  style=pozn,
  name=Poznámka,
]{poznamka}

\declaretheorem[
  style=dusl,
  name=Důsledek,
]{dusledek}

\declaretheorem[
  style=axiom,
  name=Axiom,
]{axiom}

\declaretheorem[
  style=motivace,
  name=Motivace,
]{motivace}

\DeclareMathOperator{\tg}{tg}
\DeclareMathOperator{\cotg}{cotg}
\DeclareMathOperator{\arctg}{arctg}
\DeclareMathOperator{\arccotg}{arccotg}
\DeclareMathOperator{\ord}{ord}

\title{Úvod do teorie grup \\ {\large Zápisky z přednášky doc. Mgr. Jana Šarocha. Ph.D.}}
\author{Dominik Doležel}
\date{}

\begin{document}

\maketitle

% ÚVODNÍ INFORMACE
% dobré pro poznačení informačních věcí od učitele,
% např. jeho email, jména skript apod.
%!TEX root = ../main.tex
%
\phantomsection
\addcontentsline{toc}{section}{Úvodní informace}

\begin{center}
	{\bfseries\Large Úvodní informace}\par\vspace{1em}
\end{center}

\begin{itemize}
	\item skripta: DRÁPAL, Aleš. \textit{Teorie grup: základní aspekty.} Praha: Karolinum, 2000.
	\item email: \href{mailto:saroch@karlin.mff.cuni.cz}{saroch@karlin.mff.cuni.cz}
\end{itemize}


% ZNAČENÍ
% pokud nepotřebujete, zakomentujte řádek níže
% všechny změny provádějte v souboru additional/conventions.tex
%!TEX root = ../main.tex

\phantomsection
\addcontentsline{toc}{section}{Značení}
\vspace{2em}
\begin{center}
{\bfseries\Large Značení}\par\vspace{1em}
\end{center}

Množinou přirozených čísel rozumíme množinu $\mathbb N =\left \{ 1,2,\dots \right \} $,
pak je $\mathbb N_0=\mathbb N \cup \left \{ 0 \right \} $.
Zobrazení skládáme zprava doleva, tj. jsou-li $f:A\to B$, $g: B\to C$ dvě zobrazení, pak
$g\circ f=gf:A\to C$, tj. pro $ a \in A$ je $(g\circ f)(a)=g(f(a))$. Identické zobrazení
z $A$ do $A$ značíme $\id_A$ nebo $\mathtt 1_A$.


% KAPITOLY
% \input{chapters/01_nazev_kapitoly}
% ...
\section{Operátorové grupy}
\begin{definice}
	Ať $\Omega$ je množina. Množina $G$ spolu s:
	\begin{enumerate}[$(i)$]
		\item binární operací $\cdot : G\times G\to G$ (zapisujeme infixem\footnote{mezi argumenty}),
		\item unární operací $^{-1}:G\to G$ (zapisujeme postfixem\footnote{za argumentem}),
		\item nulární operací, tj. konstantou $\mathtt 1\in G$,
		\item unárními operacemi $\omega \in \Omega: G\to G$ (zapisované prefixem\footnote{před argumentem, tady $\omega(\phantom x)$})
	\end{enumerate}
	se nazývá \textbf{$\Omega$-grupou}, pokud:
	\begin{enumerate}[$(i)$]
		\item $\cdot$ je asociativní, tj. $\forall a,b,c\in G: (a\cdot b)\cdot c = a\cdot (b\cdot c)$,
		\item $\mathtt 1$ je \textbf{neutrální prvek} vzhledem k operaci $\cdot$, tj. $\forall a \in G:a\cdot \mathtt 1 = \mathtt 1 \cdot a=a$,
		\item $\forall a \in G$ je $a^{-1}$ \textbf{inverzní prvek} k $a$, tj. $a\cdot a^{-1}=a^{-1}\cdot a=\mathtt 1$,
		\item $\forall \omega \in \Omega$ je $\omega$ \textbf{slučitelná s} operací $\cdot$, tj. $\forall a,b \in G: \omega(a\cdot b)=\omega(a)\cdot \omega(b).$
	\end{enumerate}
\end{definice}

\begin{poznamka}\,
\begin{enumerate}[i.]
	\item Je-li $\Omega = \emptyset$, pak místo o $\Omega$-grupě hovoříme jen o \textbf{grupě}.
	\item Pro všechna $a,b,c\in G$ platí:
	$$
	    (a\cdot b = a\cdot c \implies b=c) \land (b\cdot a= c\cdot a \implies b=c).
	$$
	Dokážeme aplikací $a^{-1}\cdot:$
	\begin{align*}
		a^{-1}\cdot(a\cdot b) &=a^{-1}\cdot(a\cdot c) \\
		(a^{-1}\cdot a)\cdot b &=(a^{-1}\cdot a)\cdot c \\
		\mathtt 1 \cdot b &=\mathtt 1 \cdot c.
	\end{align*}
	\item Z předchozího plyne $\left( a^{-1} \right )^{-1}=a $, neboť
	$$
	    a^{-1}\cdot \left( a^{-1} \right )^{-1} = a^{-1}\cdot a \implies a=\left( a^{-1} \right )^{-1}.
	$$
	\item Inverzní k $a\in G$ je právě jeden prvek, a sice $a^{-1}$. Neutrální prvek
	vzhledem k operaci $\cdot$ je právě jeden, a sice $\mathtt 1$. (Sporem předpokládejme, že existuje
	i $\mathtt 1^\prime\ne \mathtt 1$, ale zároveň $a\cdot \mathtt 1 = a\cdot\mathtt1^\prime\implies \mathtt 1 = \mathtt 1^\prime$,
	což je spor.)
	\item Symbol $\cdot$ se často nepíše.
\end{enumerate}
\end{poznamka}


\begin{definice}
	Ať $G$ je $\Omega$-grupa. \textbf{Řádem} $\Omega$-grupy $G$ rozumíme mohutnost množiny $G$, značíme
	$|G|$ nebo $\ord G$.
\end{definice}

\begin{definice}
	Buďte $G,H$ $\Omega$-grupy, $f:G\to H$ zobrazení. Řekneme, že $f$ je \textbf{homomorfismus}
	$\Omega$-grup $G,H$, jestliže
	\begin{enumerate}[$(i)$]
		\item $\forall a,b\in G: f(a\cdot b)=f(a)\cdot f(b)$ a
		\item $\forall a,b \in G, \forall \omega \in \Omega: f(\omega(a))=\omega(f(a)).$
\end{enumerate}
\end{definice}

\begin{lemma}\label{lemma11}
	Je-li $f:G\to H$ homomorfismus $\Omega$-grup, pak $f(\mathtt 1)= \mathtt 1$ a $\forall a\in G:f \left( a^{-1} \right ) =\left( f(a) \right )^{-1} $.
\end{lemma}

\begin{proof}
	Platí:
	$$
	    \mathtt 1 \cdot f(\mathtt 1)=f(\mathtt 1)=f(\mathtt 1\cdot \mathtt 1)=f(\mathtt 1)\cdot f(\mathtt 1)\implies f(\mathtt 1)=\mathtt 1.
	$$
	Dále
	$$
	    f(a^{-1})\cdot f(a)=f(a^{-1}\cdot a)=f(\mathtt 1)=\mathtt 1,
	$$
	ale taky
	$$
	    f(a)\cdot f(a^{-1})=f(a\cdot a^{-1})=f(\mathtt 1)=\mathtt 1,
	$$
	odkud plyne $f(a^{-1})=\left( f(a) \right )^{-1} $, jelikož jediný inverzní prvek k $f(a)$ je pouze
	$\left( f(a) \right )^{-1} $.
\end{proof}

\begin{definice}
	Ať $f: G\to H$ je homomorfismus $\Omega$-grup. Pak $f$ je:
	\begin{enumerate}[($i$)]
		\item \textbf{izomorfismus}, jestliže $f$ je bijektivní;
		\item \textbf{endomorfismus} (\textbf{grupy G}), jestliže $G=H$;
		\item \textbf{automorfismus}, jestliže je $f$ endomorfismus a izomorfismus.
	\end{enumerate}
\end{definice}

\begin{cviceni}
	Ať $f:G\to H$ je homomorfismus $\Omega$-grup. Ukažte, že $f$ je izomorfismus právě
	tehdy, když existuje homomorfismus $g:H\to G$ tak, že $f\circ g=\id_G$ a $g\circ f=\id_H$.
\end{cviceni}

\begin{lemma}
    \begin{enumerate}[($i$)]
    	\item Ať $f:G\to H$, $g:H\to K$ jsou homomorfismy $\Omega$-grup. Pak $g\circ f$ je
        homomorfismus.
        \item Je-li $f:G\to H$ homomorfismus, pak $f^{-1}:H\to G$ je opět homomorfismus.
    \end{enumerate}
\end{lemma}

\begin{proof}
    \begin{enumerate}[($i$)]
    	\item Snadné.
        \item $f^{-1}$ je jistě bijekce, ověříme jen, že $f^{-1}$ je homomorfismus. Počítejme
        $$
            f \left( f^{-1}(a)\cdot f^{-1}(b) \right )=f \left( f^{-1}(a) \right )\cdot f \left( f^{-1}(b) \right )=a\cdot b.
        $$
        Na tuto rovnost aplikujeme $f^{-1}$:
        $$
            f^{-1}(a)\cdot f^{-1}(b)=f^{-1}(a\cdot b).
        $$
        Ať $\omega \in \Omega$, $a \in H$ jsou libovolná. Pak
        $$
            f \left( \omega \left( f^{-1}(a) \right )  \right )=\omega \left( f \left( f^{-1}(a) \right )  \right )=\omega(a).
        $$
        Opět aplikujeme $f^{-1}$:
        $$
            \omega \left( f^{-1}(a) \right )=f^{-1}\left( \omega (a) \right ),
        $$
        což jsme chtěli dokázat.
        \qedhere
    \end{enumerate}
\end{proof}

\begin{definice}
	Pokud je v $\Omega$-grupě $G$ operace $\cdot$ komutativní, tj.
	$$
	    \forall a,b\in G: a\cdot b = b\cdot a,
	$$
	potom nazýváme $G$ \textbf{komutativní} (též \textbf{abelovskou}) $\Omega$-grupou.
\end{definice}

\begin{priklad}
    \begin{enumerate}[1.]
    	\item $\Omega = \emptyset$:
            \begin{itemize}
           	    \item $(\mathbb Z ,+,-,0)$ je abelovská grupa
                \item $X$ je množina, $S(X)=\left \{ \sigma:X\to X, \sigma\, \mathrm{bijekce} \right \} $\\
                s operacemi $\circ$ (skládání zobrazení), $^{-1}$ (inverzní zobrazení), $\mathtt 1$ (identické zobrazení)\\
                $S(x)$ je abelovská právě tehdy, když $|X|<3$. Je-li $X=\left \{ 1,2,\dots,n \right \},n \in \mathbb N , $
                pak $S(X):=S_n\footnote{symetrická grupa na $n$ prvcích}$.
                \item $R$ je okruh, pak $(R,+,-,0)$ je abelovská grupa,\\
                $(R^*,\cdot,^{-1},\mathtt 1)$, kde $R^*=\left \{ r\in R, \exists s \in R, r\cdot s = s\cdot r = \mathtt 1 \right \}$ je grupa
                invertibilních prvků
                \item $n\in \mathbb N , T$ je těleso, $M_n(T)$ je okruh matic $n\times n$ nad tělesem $T$\\
                $(M_n(T))^*=\left \{ A\in M_n(T), \det(A)\ne 0 \right \} := GL_n(T)\footnote{zobecněná lineární grupa}$
                \item Ať $G=(V,E)$ je neorientovaný graf. Pak $Aut(G)=\left \{ f:V\to V, f \textrm{ automorfismus grafu } G \right \} $.\\
                Speciálně pro graf $C_n$, $n \in \mathbb N \smallsetminus \left \{ 1,2 \right \} $ platí $Aut(C_n):=D_{2n}=D_n\leq S_n$.
            \end{itemize}
        \item $\Omega \ne \emptyset$:\\
            $T$ je (komutativní) těleso, $V$ je vektorový prostor nad $T$. Pak $(V,+,-,0)$ je abelovská grupa,
            $\Omega= \left \{ \cdot t:V\to V,t\in T \right \} $. $V$ je $\Omega$-grupa.\\
            Obecněji: $R$ je okruh, $M$ je (pravý) modul nad $R$. Například je-li $M=N_{2\times 3}(T)$,
            pak $\left( M,+,-,\begin{pmatrix}
                0 & 0 & 0\\
                0 & 0 & 0
            \end{pmatrix} \right ) $ je abelovská grupa. Je-li $R=M_3(T)$, $M$ je pravý $R$ modul, pak $M$
            je $\Omega$-grupa.
    \end{enumerate}
\end{priklad}

\begin{definice}
	Ať $G$ je $\Omega$-grupa, $A\subseteq G$. Pak $A$ nazveme \textbf{$\Omega$-podgrupou} $\Omega$-grupy $G$,
	píšeme $A\leq G$, pokud:
	\begin{enumerate}[($i$)]
		\item $\mathtt 1\in A$,
		\item $\forall a,b \in A:a\cdot b\in A$ a $a^{-1}\in A$,
		\item $\forall a \in A:\forall \omega \in \Omega: \omega(a)\in A$.
    \end{enumerate}
\end{definice}

\begin{dusledek}
	Ať $G$ je $\Omega$-grupa. Množina $Aut(G)$ všech automorfismů $\Omega$-grupy $G$ tvoří spolu s
	operacemi $\circ, ^{-1},\id_G$ grupu. Platí $Aut(G)\leq S(G).$
\end{dusledek}

\begin{dusledek}
	Ať $G$ je $\Omega$-grupa, $\omega\in \Omega$. Pak $\omega:G\to G$ je endomorfismus grupy $G$
	(tj. $\emptyset$-grupy $G$). Mj. platí, že $\omega(\mathtt 1)=\mathtt 1$, $\omega \left( a^{-1} \right ) = \left( \omega(a) \right )^{-1} \forall a \in G.$
\end{dusledek}

\begin{proof}
	Plyne ihned z \ref{lemma11}.
\end{proof}

\begin{poznamka}
	Často je přímo $\Omega \subseteq End(G)=\left \{ f:G\to G,f \textrm{ je endomorfismus grupy }G \right \} $.
\end{poznamka}

\begin{lemma}
	Ať $G$ je grupa, $g\in G$ libovolné. Označme $\alpha_G:G\to G$ takové, že $\forall a \in G:\alpha_G (a)=gag^{-1}$.
	Pak je $\alpha_g\in Aut(G)$ a nazývá se vnitřní automorfismus určený prvkem $g$.
\end{lemma}


% \listoffigures

\backmatter

\end{document}
