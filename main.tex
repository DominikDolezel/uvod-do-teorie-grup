\documentclass[11pt]{template/cauchy}
\usepackage[czech]{babel}

% IMPORTY
\usepackage{enumerate}
\usepackage{mlmodern}
\usepackage{template/mathsphystools}
\usepackage{amsthm,thmtools,xcolor,amsmath,amssymb, mathtools}
\usepackage[scr]{rsfso}
% \usepackage{thmstyles}
\usepackage{tabularx}
\usepackage{graphicx}
\usepackage{multicol}
\usepackage{appendix}
\usepackage{tabularx}
\usepackage{subcaption}
\usepackage{hyperref}
\usepackage{comment}
\usepackage{lscape}
\usepackage{xcolor}
\usepackage{wrapfig}
\usepackage{centernot}
\usepackage{pgfplots}
\usepackage{caption}
\graphicspath{{images/}}
\pgfplotsset{width=10cm,compat=1.9}

% \usepackage{tikzit}
% \input{default.tikzstyles}
% je potřeba přiložit soubor default.tikzstyles


% STYLY
% definice, příklad, cvičení, věta, lemma, řešení, poznámka, důsledek, axiom, motivace

\declaretheoremstyle[
  headfont=\color{red}\normalfont\bfseries,
  bodyfont=\color{black}\normalfont,
]{def}

\declaretheoremstyle[
  headfont=\color{blue}\normalfont\bfseries,
  bodyfont=\color{black}\normalfont,
]{pr}

\declaretheoremstyle[
  headfont=\color{teal}\normalfont\bfseries,
  bodyfont=\color{black}\normalfont\itshape,
]{cv}

\declaretheoremstyle[
  headfont=\color{green}\normalfont\bfseries,
  bodyfont=\color{black}\normalfont\itshape,
]{veta}

\declaretheoremstyle[
  headfont=\color{lime}\normalfont\bfseries,
  bodyfont=\color{black}\normalfont\itshape,
]{lemma}

\declaretheoremstyle[
  headfont=\color{black}\normalfont\itshape,
  bodyfont=\color{black}\normalfont,
  numbered=no
]{res}

\declaretheoremstyle[
  headfont=\color{brown}\normalfont\bfseries,
  bodyfont=\color{black}\normalfont,
]{pozn}

\declaretheoremstyle[
  headfont=\color{brown}\normalfont\bfseries,
  bodyfont=\color{black}\normalfont,
]{dusl}

\declaretheoremstyle[
  headfont=\color{cyan}\normalfont\bfseries,
  bodyfont=\color{black}\normalfont,
]{axiom}

\declaretheoremstyle[
  headfont=\color{black}\normalfont\bfseries,
  bodyfont=\color{black}\normalfont,
]{motivace}

\declaretheorem[
  style=def,
  name=Definice,
]{definice}

\declaretheorem[
  style=pr,
  name=Příklad,
]{priklad}

\declaretheorem[
  style=cv,
  name=Cvičení,
]{cviceni}

\declaretheorem[
  style=res,
  name=Řešení,
]{reseni}

\declaretheorem[
  style=veta,
  name=Věta,
]{veta}

\declaretheorem[
  style=lemma,
  name=Lemma,
]{lemma}

\declaretheorem[
  style=pozn,
  name=Poznámka,
]{poznamka}

\declaretheorem[
  style=dusl,
  name=Důsledek,
]{dusledek}

\declaretheorem[
  style=axiom,
  name=Axiom,
]{axiom}

\declaretheorem[
  style=motivace,
  name=Motivace,
]{motivace}

\DeclareMathOperator{\tg}{tg}
\DeclareMathOperator{\cotg}{cotg}
\DeclareMathOperator{\arctg}{arctg}
\DeclareMathOperator{\arccotg}{arccotg}

\title{Úvod do sem patří název \\ {\large Zápisky z přednášky Jméno Příjmení učitele}}
\author{Jméno Příjmení}
\date{}

\begin{document}

\maketitle

% ÚVODNÍ INFORMACE
% dobré pro poznačení informačních věcí od učitele,
% např. jeho email, jména skript apod.
%!TEX root = ../main.tex
%
\phantomsection
\addcontentsline{toc}{section}{Úvodní informace}

\begin{center}
	{\bfseries\Large Úvodní informace}\par\vspace{1em}
\end{center}

\begin{itemize}
	\item skripta: DRÁPAL, Aleš. \textit{Teorie grup: základní aspekty.} Praha: Karolinum, 2000.
	\item email: \href{mailto:saroch@karlin.mff.cuni.cz}{saroch@karlin.mff.cuni.cz}
\end{itemize}


% ZNAČENÍ
% pokud nepotřebujete, zakomentujte řádek níže
% všechny změny provádějte v souboru additional/conventions.tex
%!TEX root = ../main.tex

\phantomsection
\addcontentsline{toc}{section}{Značení}
\vspace{2em}
\begin{center}
{\bfseries\Large Značení}\par\vspace{1em}
\end{center}

Množinou přirozených čísel rozumíme množinu $\mathbb N =\left \{ 1,2,\dots \right \} $,
pak je $\mathbb N_0=\mathbb N \cup \left \{ 0 \right \} $.
Zobrazení skládáme zprava doleva, tj. jsou-li $f:A\to B$, $g: B\to C$ dvě zobrazení, pak
$g\circ f=gf:A\to C$, tj. pro $ a \in A$ je $(g\circ f)(a)=g(f(a))$. Identické zobrazení
z $A$ do $A$ značíme $\id_A$ nebo $\mathtt 1_A$.


% KAPITOLY
% \input{chapters/01_nazev_kapitoly}
% ...

% \listoffigures

\backmatter

\end{document}
