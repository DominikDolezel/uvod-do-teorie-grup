\section{Operátorové grupy}
\begin{definice}
	Ať $\Omega$ je množina. Množina $G$ spolu s:
	\begin{enumerate}[$(i)$]
		\item binární operací $\cdot : G\times G\to G$ (zapisujeme infixem\footnote{mezi argumenty}),
		\item unární operací $^{-1}:G\to G$ (zapisujeme postfixem\footnote{za argumentem}),
		\item nulární operací, tj. konstantou $\mathtt 1\in G$,
		\item unárními operacemi $\omega \in \Omega: G\to G$ (zapisované prefixem\footnote{před argumentem, tady $\omega(\phantom x)$})
	\end{enumerate}
	se nazývá \textbf{$\Omega$-grupou}, pokud:
	\begin{enumerate}[$(i)$]
		\item $\cdot$ je asociativní, tj. $\forall a,b,c\in G: (a\cdot b)\cdot c = a\cdot (b\cdot c)$,
		\item $\mathtt 1$ je \textbf{neutrální prvek} vzhledem k operaci $\cdot$, tj. $\forall a \in G:a\cdot \mathtt 1 = \mathtt 1 \cdot a=a$,
		\item $\forall a \in G$ je $a^{-1}$ \textbf{inverzní prvek} k $a$, tj. $a\cdot a^{-1}=a^{-1}\cdot a=\mathtt 1$,
		\item $\forall \omega \in \Omega$ je $\omega$ \textbf{slučitelná s} operací $\cdot$, tj. $\forall a,b \in G: \omega(a\cdot b)=\omega(a)\cdot \omega(b).$
	\end{enumerate}
\end{definice}

\begin{poznamka}\,
\begin{enumerate}[i.]
	\item Je-li $\Omega = \emptyset$, pak místo o $\Omega$-grupě hovoříme jen o \textbf{grupě}.
	\item Pro všechna $a,b,c\in G$ platí:
	$$
	    (a\cdot b = a\cdot c \implies b=c) \land (b\cdot a= c\cdot a \implies b=c).
	$$
	Dokážeme aplikací $a^{-1}\cdot:$
	\begin{align*}
		a^{-1}\cdot(a\cdot b) &=a^{-1}\cdot(a\cdot c) \\
		(a^{-1}\cdot a)\cdot b &=(a^{-1}\cdot a)\cdot c \\
		\mathtt 1 \cdot b &=\mathtt 1 \cdot c.
	\end{align*}
	\item Z předchozího plyne $\left( a^{-1} \right )^{-1}=a $, neboť
	$$
	    a^{-1}\cdot \left( a^{-1} \right )^{-1} = a^{-1}\cdot a \implies a=\left( a^{-1} \right )^{-1}.
	$$
	\item Inverzní k $a\in G$ je právě jeden prvek, a sice $a^{-1}$. neutrální prvek
	vzhledem k operaci $\cdot$ je právě jeden, a sice $\mathtt 1$. (Sporem předpokládejme, že existuje
	i $\mathtt 1^\prime\ne \mathtt 1$, ale zároveň $a\cdot \mathtt 1 = a\cdot\mathtt1^\prime\implies \mathtt 1 = \mathtt 1^\prime$,
	což je spor.)
	\item Symbol $\cdot$ se často nepíše.
\end{enumerate}
\end{poznamka}


\begin{definice}
	Ať $G$ je $\Omega$-grupa. \textbf{Řádem} $\Omega$-grupy $G$ rozumíme mohutnost množiny $G$, značíme
	$|G|$ nebo $\ord G$.
\end{definice}

\begin{definice}
	Buďte $G,H$ $\Omega$-grupy, $f:G\to H$ zobrazení. Řekneme, že $f$ je \textbf{homomorfismus}
	$\Omega$-grup $G,H$, jestliže
	\begin{enumerate}[$(i)$]
		\item $\forall a,b\in G: f(a\cdot b)=f(a)\cdot f(b)$ a
		\item $\forall a,b \in G, \forall \omega in \Omega: f(\omega(a))=\omega(f(a)).$
\end{enumerate}
\end{definice}

\begin{lemma}
	Je-li $f:G\to H$ homomorfismus $\Omega$-grup, pak $f(\mathtt 1)= \mathtt 1$ a $\forall a\in G:f \left( a^{-1} \right ) =\left( f(a) \right )^{-1} $.
\end{lemma}

\begin{proof}
	\,
\end{proof}
